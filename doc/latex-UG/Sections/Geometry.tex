\section{Geometry}

Several ways to handle geometries are implemented. One can import a pre-existing geometry from a file, generate cubic and ellipsoidal geometries from scratch, or modify existing geometries.\\


\subsection{Import a geometry}

One can import a {\tt geometry.structure} from a STL or a STEP file using the {\tt importGeometry} \textsc{Matlab} built-in function. User must however give the number of cavities and the IDs of the faces exposed to radiation.\\

\textbf{Example :}\\

\begin{lstlisting}[language=Matlab]
geometry.structure = tools.importGeometry('Torus.stl');
geometry.exposed_faces = 6;
geometry.nbr_cavities = 5;
\end{lstlisting}
\ 

\subsection{Generate a new geometry}

One can generate new cubic or ellipsoidal geometries. The function {\tt generateGeometry} returns a {\tt geometry} structure, with the following arguments : 

\renewcommand{\arraystretch}{1.5}
\begin{table}[h]
    \centering
    \begin{tabular}{|>{\customfont}c|>{\customfont}c|>{\customfont}c|}
        \hline 
        \rowcolor{gray!30}
        \textbf{Fields} & \textbf{Type} & \textbf{Description} \\ \hline
        \tt shape & \tt string & {\tt 'cube'} or {\tt 'ellipsoid'} \\ \hline 
        \tt unit & \tt array (double) & semi-axes of the ellipsoid or edge of the cube $(m)$ \\ \hline 
        \tt nbr\_cavities	& \tt array (double) & Number of cavities that must be digged \\ \hline 
        \tt type & \tt string & Type of cavities; must be {\tt 'full'} or {\tt 'void'} \\ \hline
        \tt min\_radius & \tt array (double) & Minimum radius of a cavity $(m)$ \\ \hline 
        \tt max\_radius & \tt array (double) & Maximum radius of a cavity $(m)$ \\ \hline 
    \end{tabular}
\end{table}
\ 

The radius of the spherical heterogenous cavities is randomized between {\tt min\_radius} and {\tt max\_radius}. Set {\tt nbr\_cavities} to {\tt 0} for homogeneous geometry.\\

\textbf{Examples :}\\

\begin{lstlisting}[language=Matlab]
c_geometry = tools.generateGeometry('cube',15,100,'void',0.2,1.2); 
e_geometry = tools.generateGeometry('ellipsoid',[1,0.5,0.5],5,'full',0.05,0.11); 
\end{lstlisting}
\ 


\subsection{Modify an existing geometry}

One can modify a pre-existing entirely filled geometry to insert cavities or heterogeneity using the {\tt modifyGeometry} function, that takes the following parameters : 

\renewcommand{\arraystretch}{1.5}
\begin{table}[H]
    \centering
    \begin{tabular}{|>{\customfont}c|>{\customfont}c|>{\customfont}c|}
        \hline 
        \rowcolor{gray!30}
        \textbf{Fields} & \textbf{Type} & \textbf{Description} \\ \hline
        \tt previous\_geometry & \tt pde.DiscreteGeometry & Geometry to modify \\ \hline
        \tt shape & \tt string & {\tt 'cube'} or {\tt 'ellipsoid'} \\ \hline 
        \tt unit & \tt array (double) & semi-axes of the ellipsoid or edge of the cube $(m)$ \\ \hline 
        \tt nbr\_cavities	& \tt array (double) & Number of cavities that must be digged \\ \hline 
        \tt type & \tt string & Type of cavities; must be {\tt 'full'} or {\tt 'void'} \\ \hline
        \tt min\_radius & \tt array (double) & Minimum radius of a cavity $(m)$ \\ \hline 
        \tt max\_radius & \tt array (double) & Maximum radius of a cavity $(m)$ \\ \hline 
    \end{tabular}
\end{table}
\ 

\textbf{Examples :}\\

\begin{lstlisting}[language=Matlab]
cube = tools.importGeometry('Cube.stl');
c_geometry = tools.modifyGeometry(cube, 'cube', 2, 30, 'full', 0.1, 0.9);
ellipsoid = tools.importGeometry('Ellipsoid.stl');
e_geometry = tools.modifyGeometry(ellipsoid, 'ellipsoid',[1,0.5,0.5], 30, 'void', 0.05, 0.11);
\end{lstlisting}

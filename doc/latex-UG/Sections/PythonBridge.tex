\section{Python Bridge}

\subsection{MATLAB Engine : {\tt matlabengine}}

By calling the matlabengine package, you can directly start a \textsc{Matlab} session and call seamlessly call the model functions.

For more information and advanced use, please consult the official documentation.

\subsubsection{Installation}

\textbf{Prerequisites}\\

Python $<=$ 3.11

\textsc{Matlab} Desktop with the PDE Toolbox installed

\begin{lstlisting}[language=sh]
$ pip install matlabengine
\end{lstlisting}

\subsubsection{Starting the engine}

One need to start or connect to a \textsc{Matlab} session before doing anything.

\begin{lstlisting}[language=Python]
eng = matlab.engine.start_matlab()
\end{lstlisting}

The \textsc{Matlab} Engine should detect the appropriate version automatically but in some edge cases, it could not work. In this case, please consult the official documentation on the subject.

\subsubsection{Calling Functions}

Calling the model functions with the \textsc{Matlab} Engine is simple : it is the exact same way as you would do in \textsc{Matlab}. The only difference is that you need to add the {\tt eng} prefix when calling them. For example :

\begin{lstlisting}[language=Python]
geo = eng.tools.generateGeometry('ellipsoid',matlab.double([1.0,0.5,0.5]),3.0,0.1,0.3)
\end{lstlisting}

\medskip 

\textbf{Be careful !}\\

When calling a function that does not return one value, you need to add the {\tt nargout} parameter to indicate how many objects will be returned.

\subsubsection{Handling data types}

\textbf{Arrays and Integerss}\\

\textsc{Python} and \textsc{Matlab} data types are not equivalent : therefore {\tt matlabengine} translates them.

The majority of this model functions are expecting \textsc{Matlab} '{\tt double}' array type. But \textsc{Python} '{\tt int}' type is translated as a \textsc{Matlab} '{\tt int64}' type, leading to errors.

Hence, you need to be sure to pass '{\tt float}' types to the MATLAB engine functions. \textsc{Python} '{\tt float}' is the only type converted as a \textsc{Matlab} '{\tt double}'\\

\textbf{Structures}\\

To create a \textsc{Matlab} structure ('{\tt geometry}', '{\tt options}', '{\tt material}'), you need to pass a \textsc{Python} dictionnary containing fields name ('{\tt string}') as keys and the data to transmit as values. For example:

\begin{lstlisting}[language=Python]
basalt = {'rho': 1000.0,"cp": 100.0,"T_0": 2000.0}
\end{lstlisting}

\textbf{Extract from the official documentation}\\

\begin{table}[h]
    \centering
    \begin{tabular}{|>{\customfont}c|>{\customfont}c|>{\customfont}c|}
        \hline 
        \rowcolor{gray!30}
        \textbf{Python type} & \textbf{MATLAB Type} \\ \hline
        \tt float & \tt double \\ \hline
        \tt complex & complex \tt double \\ \hline 
        \tt int & \tt int64 \\ \hline 
        \tt float(nan) & \tt NaN \\ \hline 
        \tt float(inf) & \tt inf \\ \hline
        \tt bool  & \tt logical \\ \hline 
        \tt str   & \tt char \\ \hline 
        \tt dict  & \tt structure \\ \hline 
        \tt list  & \tt cell array \\ \hline 
        \tt set   & \tt cell array \\ \hline 
        \tt tuple & \tt cell array \\ \hline 
    \end{tabular}
\end{table}
\ 
\subsubsection{API Documentation}

From the official documentation.\\

\textbf{matlab.double}\\

A lot of \textsc{Matlab} built in functions need arrays to work properly. This function allows a python integer, a float or a list to be converted to a \textsc{Matlab} `double` array

\begin{table}[H]
    \centering
    \begin{tabular}{|>{\customfont}c|>{\customfont}c|>{\customfont}c|}
        \hline 
        \rowcolor{gray!30}
        \textbf{Fields} & \textbf{Type} & \textbf{Description}\\ \hline
        {\tt struct} & {\tt int}, {\tt list} or {\tt dict} & structure to convert \\ \hline
    \end{tabular}
\end{table}
\

\textbf{eng.workspace}\\

To make \textsc{Matlab} consider variables (and save them), you need to add them to the \textsc{Matlab} Workspace. In \textsc{Python}, the \textsc{Matlab} Workspace is represented as a dictionary, you can add variables to the Workspace the same way you would add a new field to a \textsc{Python} dictionnary. For example:

\begin{lstlisting}[language=Python]
eng.workspace['results']=results
\end{lstlisting}
\

\textbf{eng.load}\\

Load \textsc{Matlab} variables from a given {\tt `filepath`}. Can only import  {\tt `.mat`} files.

\begin{table}[ht]
    \centering
    \begin{tabular}{|>{\customfont}c|>{\customfont}c|>{\customfont}c|}
        \hline 
        \rowcolor{gray!30}
        \textbf{Fields} & \textbf{Type} & \textbf{Description}\\ \hline
        {\tt filepath} & {\tt string} & file to load \\ \hline
    \end{tabular}
\end{table}
\

\textbf{eng.save}\\

Used to save \textsc{Matlab} workspace variables under a given {\tt `filepath`}. Output file can only be saved under a {\tt `.mat`} extension.

\begin{table}[ht]
    \centering
    \begin{tabular}{|>{\customfont}c|>{\customfont}c|>{\customfont}c|}
        \hline 
        \rowcolor{gray!30}
        \textbf{Fields} & \textbf{Type} & \textbf{Description}\\ \hline
        {\tt filepath} & {\tt string} & file under which to save the variable \\ \hline
        {\tt var} & {\tt string} & \textsc{Matlab} Workspace variable name \\ \hline
    \end{tabular}
\end{table}
\

Example :

\begin{lstlisting}[language=Python]
eng.save("test.mat", "results", nargout=0)
\end{lstlisting}

\subsection{SciPy Integration}

\textbf{Load MAT Files}\\

\textsc{Matlab} {\tt .mat} files can be loaded in \textsc{Python} using {\tt scipy.io.loadmat}. They are then loaded as dictionnaries containing {\tt numpy.ndarray}\\

Example :

\begin{lstlisting}[language=Python]
results = scipy.io.loadmat("file.mat")
\end{lstlisting}
\section{Simulate} 

The function first defines the time list to solve the transient problem. 

\begin{lstlisting}[language=Matlab]
tlist = 0:options.dt:options.tmax;
\end{lstlisting}

The code checks whether there's any latent heat property. If so, adds a gaussian function with $\mu = 1373$, $\sigma = 5$ to the specific heat capacity value. It then reflects the latent heat release. \\

The temperature-dependent properties depending on the cavities type (\texttt{full} or \texttt{void}) are given. For the main material for example,

\begin{lstlisting}[language=Matlab]
if options.material.lambda == "dresser"
  lambda_material = @(location, state) 0.35+0.85*exp(-1.7e-3*(state.u-273)); 
elseif options.material.lambda == "hebei"
  lambda_material = @(location, state) 0.59+0.23*exp(-4.3e-4*(state.u-273)); 
elseif options.material.lambda == "tholeiitic"
  lambda_material = @(location, state) 0.46+0.95*exp(-2.3e-3*(state.u-273)); 
else 
  error('Unknow material type.');
end
\end{lstlisting}

\medskip
Create a PDE model for transient thermal analysis.

\begin{lstlisting}[language=Matlab]
thermalModel = createpde('thermal','transient');
\end{lstlisting}

\medskip
Import the geometry. 

\begin{lstlisting}[language=Matlab]
thermalModel.Geometry=geometry.structure;
\end{lstlisting}

\medskip
Generate a mesh for the geometry stored in the model object. Parameter \texttt{Hmax} is the maximum mesh edge length, and geometric order must be specified as \texttt{linear} or \texttt{quadratic}.

\begin{lstlisting}[language=Matlab]
generateMesh(thermalModel,'Hmax',0.2,'GeometricOrder','quadratic');\end{lstlisting}

\medskip
Specify the thermal properties of the main material located in cell 1, and the initial temperature 

\begin{lstlisting}[language=Matlab]
thermalProperties(thermalModel,'Cell',1,'ThermalConductivity',lambda, ...
                  'MassDensity',options.material.rho, ...
                  'SpecificHeat',options.material.cp);
thermalIC(thermalModel,options.material.T_0,'Cell',1);
\end{lstlisting}

\medskip
If the cavities are not voids but heteregenous material, set the material properties and inital temperature

\begin{lstlisting}[language=Matlab]
thermalProperties(thermalModel,'Cell',2:geometry.nbr_cavities+1, ...
                  'ThermalConductivity',lambda, ...
                  'MassDensity',options.cavities_material.rho, ...
                  'SpecificHeat',options.cavities_material.cp);
thermalIC(thermalModel,options.cavities_material.T_0,'Cell',2:geometry.nbr_cavities+1);
\end{lstlisting}

\medskip
Define the radiation boundary condition on the geometry surface of the model.

\begin{lstlisting}[language=Matlab]
thermalModel.StefanBoltzmannConstant = 5.670373E-8;
thermalBC(thermalModel,'Face',geometry.exposed_faces, ...
          'Emissivity',@(region,state) options.eps, ...
          'AmbientTemperature',options.T_out, ...
          'Vectorized','on');
\end{lstlisting}

\medskip
Solve the model for the time steps in \texttt{tlist}.

\begin{lstlisting}[language=Matlab]
Results = solve(thermalModel,tlist);
\end{lstlisting}

\medskip
The results are stored in a structure containing the following properties : 
\begin{enumerate}
    \item \textbf{Temperature}. Temperature values at nodes.
    \item \textbf{SolutionTimes}. Same list as the tlist input to solve.
    \item \textbf{XGradients}. $x$-component of the temperature gradient at nodes.
    \item \textbf{YGradients}. $y$-component of the temperature gradient at nodes.
    \item \textbf{ZGradients}. $z$-component of the temperature gradient at nodes.
    \item \textbf{Mesh}. Finite element mesh, which is also a structure containing nodes coordinates, elements, maximum and minimum element size, and geometric order.
\end{enumerate}